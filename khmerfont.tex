
\usepackage[sumlimits]{amsmath}  
\usepackage{amssymb,amsthm} 
\usepackage{xltxtra}  
\usepackage{titlesec} 
\usepackage{titletoc} 
\usepackage{hyperref} 
\usepackage[title,page,titletoc]{appendix} 

%font color=========================================
\usepackage[x11names,svgnames,dvipsnames]{xcolor}
\everymath{\displaystyle}
\usepackage[cal=boondoxo]{mathalfa} % mathcal
\usepackage{mathtools}
\usepackage{amsfonts}
\usepackage{amsthm}
\usepackage{palatino}
\usepackage{mathpazo}
\usepackage{tikz}%for draw
\usepackage{wrapfig}
\usepackage{pdfpages}
\usetikzlibrary{mindmap} % LATEX and plain TEX
\usepgflibrary{decorations.text} % LATEX and plain TEX and pure pgf 
\usepgflibrary[decorations.text] % ConTEXt and pure pgf 
\usetikzlibrary{decorations.text} % LATEX and plain TEX when using TikZ 
\usetikzlibrary[decorations.text] % ConTEXt when using TikZ
\usetikzlibrary{automata} % LATEX and plain TEX
\usetikzlibrary[automata] % ConTEXt
\usepackage{fontspec}
%\usepackage{minted}
%===============================
\defaultfontfeatures{Script=khmr,Mapping=tex-text} % to support TeX conventions like ``---'' 
\newfontscript{Khmer}{khmr} 
\usepackage{xunicode}  
\newcommand{\KhOS}{\fontspec[Script=Khmer,Scale=0.92]{Khmer OS}\selectfont} 
\newcommand{\KhOSML}{\fontspec[Script=Khmer]{Khmer OS Muol Light} \selectfont} 
%\newcommand{\KhMEF}{\fontspec[Script=Khmer]{Khmer MEF2} \selectfont} 

\setmainfont[Script=Khmer,Scale=0.95]{Khmer OS Battambang} 
%%%%%%% This macro is to produce khmer numbering by adopting the thai numbering method
\makeatletter  
\def\@khmernum#1{\expandafter\@@khmernum\number#1\@nil}  
\def\@@khmernum#1{%  
  \ifx#1\@nil  
  \else  
  \char\numexpr#1+"17E0\relax  
  \expandafter\@@khmernum\fi  
}  
%\def\khmercounter#1{\expandafter\@khmernum\csname c@#1\endcsname}  
\renewcommand\@arabic{\@khmernum} % to reset number in \arabic to \khmernum  
\makeatother  

\theoremstyle{plain}  
\newtheorem{theorem}{\KhOSML ទ្រឹស្ដីបទ}[chapter]   
\newtheorem{proposition}{\KhOSML សំណើ}[theorem]  
\newtheorem{corollary}{\KhOSML កូរ៉ូលែរ}[theorem] 

\theoremstyle{definition} 
\newtheorem{definition}{\KhOSML និយមន័យ}[chapter]%
\newtheorem{lemma}{\KhOSML ឡែម}[definition]%

\theoremstyle{remark} 
\newtheorem{remark}{\KhOSML សម្គាល់}%[theorem] 
\newtheorem{example}{\KhOSML  ឧទាហរណ៍}[chapter] 
\newtheorem{exercise}{\KhOSML លំហាត់}[chapter] 
\renewcommand{\figurename}{\KhOSML  រូប} 
\renewcommand{\proofname}{\KhOSML សម្រាយបញ្ជាក់} 

\renewcommand{\contentsname}{មាតិកា} 
\renewcommand{\indexname}{លិបិក្រម} 
\renewcommand{\chaptername}{ជំពូកទី} 
\renewcommand{\partname}{ផ្នែកទី} 
\renewcommand{\appendixname}{ជំពូកបន្ថែម} 
\renewcommand{\listfigurename}{បញ្ជីរូបភាព} 
\renewcommand{\listtablename}{បញ្ជីតារាង} 
\renewcommand{\appendixpagename}{\KhOSML ជំពូកបន្ថែម} 

%%%% to format part, chapter, section, subsection and subsubsection using titlesec package

\titleformat{\section}[block] 
{\large \KhOSML}{{\large\thesection}}{1em}{}

\titleformat{\subsection}[block] 
{\normalsize \KhOSML}{{\normalsize\thesubsection}}{1em}{}

\titleformat{\subsubsection}[block] 
{\normalsize\KhOSML }{{\normalsize\thesubsubsection}}{1em}{}

\titleformat{\chapter}[display] 
    {\Huge\KhOSML }{\filright\LARGE\normalfont\chaptername\ \thechapter}{2ex}         %sep
    {\filcenter} 

\titleformat{\part}[display] 
{\filcenter\LARGE\KhOSML}{{\large\partname\; \thepart}}{1em}{\thispagestyle{empty}}

\makeatletter 
%%% to set khmer calendar 
\renewcommand\today{\@khmernum\day\space-\space\ifcase\month\or 
  មករា\or កុម្ភៈ\or មិនា\or មេសា\or ឧសភា\or មិថុនា\or
  កក្កដា\or សីហា\or កញ្ញា\or តុលា\or វិច្ឆិកា\or ធ្នូ\fi
 \space-\space\@khmernum\year} 
 \makeatother 
%%%%%%%%%%


\title{\KhOSML Machine Learning} 
\author{ ថុល​ ចាន់ថន}
\date{\today}

\XeTeXlinebreaklocale "khm" % allow line breaks 
\XeTeXlinebreakskip = 0pt plus 1pt minus 1pt % for line breakskip suitable for Khmer Unicode fonts 
%=============================design=============================